% !TeX TS-program = XeLaTeX
\documentclass[a4paper,12pt]{article}

% polyglossia should go first!
\usepackage{polyglossia} % multi-language support
\setmainlanguage{russian}
\setotherlanguage{english}

\usepackage{amsmath} % math symbols, new environments and stuff
\usepackage{unicode-math} % for changing math font and unicode symbols
\usepackage[style=english]{csquotes} % fancy quoting
\usepackage{microtype} % for better font rendering
\usepackage[backend=biber]{biblatex} % for bibliography
\usepackage{hyperref} % for refs and URLs
\usepackage{graphicx} % for images (and title page)
\usepackage{geometry} % for margins in title page
\usepackage{tabu} % for tabulars (and title page)
\usepackage{placeins} % for float barriers
\usepackage{titlesec} % for section break hooks
%\usepackage[justification=centering]{caption} % for forced captions centering
%\usepackage{subcaption} % for subfloats
%\usepackage{rotating} % for rotated labels in tables
%\usepackage{tikz} % for TiKZ
%\usepackage{dot2texi} % for inline dot graphs
\usepackage{listings} % for listings 
\usepackage{algpseudocode} % for good-looking pseudo-code
%\usepackage{upquote} % for good-looking quotes in source code (used for custom languages)
%\usepackage{multirow} % for multirow cells in tabulars
%\usepackage{afterpage} % for nice landspace floats and longtabus
%\usepackage{pdflscape} % for landspace orientation
%\usepackage{xcolor} % colors!
%\usepackage{enumitem} % for unboxed description labels (long ones)
%\usepackage{numprint} % pretty-printing numbers
%\usepackage{longtable} % for longtabu

\defaultfontfeatures{Mapping=tex-text} % for converting "--" and "---"
\setmainfont{CMU Serif}
\setsansfont{CMU Sans Serif}
\setmonofont{CMU Typewriter Text}
\setmathfont{XITS Math}
\DeclareSymbolFont{letters}{\encodingdefault}{\rmdefault}{m}{it} % for Russian in math
%\MakeOuterQuote{"} % enable auto-quotation

% new page and barrier after section, also phantom section after clearpage for
% hyperref to get right page.
% clearpage also outputs all active floats:
%\newcommand{\sectionbreak}{\FloatBarrier\newpage\phantomsection}
%\newcommand{\subsectionbreak}{\FloatBarrier}
\renewcommand{\thesection}{\arabic{section}} % no chapters
\numberwithin{equation}{section}
\renewcommand{\figurename}{Рисунок} % Russian standards
%\usetikzlibrary{shapes,arrows,trees}

\lstset{
  numbers=left,
  numberstyle=\scriptsize,
  basicstyle=\ttfamily\scriptsize,
  columns=fullflexible,
  keepspaces, % for spaces in unicode text!
  captionpos=t % Russian standards, again
}
\renewcommand{\lstlistingname}{Листинг}

% don't ask. please. just don't
\algrenewcommand{\algorithmicrequire}{\textbf{Ввод:}}
\algrenewcommand{\algorithmicensure}{\textbf{Вывод:}}
\algrenewcommand{\algorithmicif}{\textbf{если}}
\algrenewcommand{\algorithmicthen}{\textbf{то}}
\algrenewcommand{\algorithmicelse}{\textbf{иначе}}
\algrenewcommand{\algorithmicfor}{\textbf{по}}
\algrenewcommand{\algorithmicforall}{\textbf{по всем}}
\algrenewcommand{\algorithmicdo}{\textbf{повторять}}
\algrenewcommand{\algorithmicwhile}{\textbf{пока}}
\algrenewcommand{\algorithmicloop}{\textbf{повторять}}
\algrenewcommand{\algorithmicrepeat}{\textbf{повторять}}
\algrenewcommand{\algorithmicuntil}{\textbf{пока не}}
\algrenewcommand{\algorithmicreturn}{\textbf{вернуть}}
\algtext*{EndWhile}
\algtext*{EndFor}
\algtext*{EndLoop}
\algtext*{EndIf}
\algtext*{EndProcedure}
\algtext*{EndFunction}

\newcommand{\un}[1]{\: \mathit{#1}} % Unit measurements

\title{Анализ вредных и опасных факторов}
\author{Амиантов Н.И., ИУ7-82}
\date{\today}

\makeatletter
\let\thetitle\@title
\let\theauthor\@author
\let\thedate\@date
\makeatother

\begin{document}

\section{Анализ вредных и опасных факторов}

\subsection{Микроклимат}


\section{Постановка задачи}

Для начала опишем задачу о назначениях.

\subsection{Содержательная постановка}
В распоряжении работодателя имеется $n$ работ и $n$ исполнителей. Стоимость выполнения $i$-ой работы $j$-ым исполнителем составляет $C_{ij}$ единиц. Требуется распределить работы между исполнителями так, чтобы:
\begin{itemize}
\item Каждый исполнитель выполнял ровно одну работу;
\item Общая стоимость выполнения работ была минимальна.
\end{itemize}

\subsection{Математическая постановка}

Зададим матрицу $X$ размером $n*n$, заполненную по правилу:

\[
X_{ij} = \begin{cases}
  1 \text{ -- \textit{если $i$-ая работа выполняется $j$-ым исполнителем,}} \\
  0 \text{ -- \textit{иначе}} \\
\end{cases}
\]

Тогда общая стоимость выполненных работ:

\[ f = \sum_{i=1}^n \sum_{j=1}^n C_{ij} X_{ij} \]

Добавим условия: условие того, что $i$-ую работу выполняет один исполнитель:

\[ \sum_{j=1}^n X_{ij} = 1, i=\overline{1:n} \]

Условие того, что $j$-ый сотрудник выполняет одну работу:

\[ \sum_{i=1}^n X_{ij} = 1, j=\overline{1:n} \]

Решим задачу нахождения максимума функции $f$. Решаемая задача:

\[
\begin{cases}
  f = \sum_{i=1}^n \sum_{j=1}^n C_{ij} X_{ij} \rightarrow max \\
  \sum_{j=1}^n X_{ij} = 1, i=\overline{1:n} \\
  \sum_{i=1}^n X_{ij} = 1, j=\overline{1:n} \\
  X_{ij} \in {0, 1}
\end{cases}
\]

В случае нахождения минимума функции задача приводится к задаче нахождения максимума:

\begin{align*}
  g &= -f = \sum_{i=1}^n \sum_{j=1}^n C_{ij} X_{ij} \rightarrow min \\
  f &= -\sum_{i=1}^n \sum_{j=1}^n C_{ij} X_{ij} \rightarrow max
\end{align*}

Для решения подобной задачи подходит венгерский метод.

\section{Описание метода}

Даётся описание алгоритма в псевдокодах:

\begin{algorithmic}
  \Require $M$ -- матрица размером $n*n$, описывающая стоимости операций.
  \Ensure назначения

  \Comment{Первый этап}
  \ForAll{столбцам $M$} \\ 
    вычесть $a$ -- минимальный из столбца
  \EndFor
  \ForAll{строкам $M$}
    вычесть $b$ -- минимальный из строки
  \EndFor
  
  \Comment{Основной цикл}
  \While{процесс не завершён}
    \ForAll{$0 \in M$}
      \If{в строке с ним $\not\exists \: 0^*$}
        пометить как $0^*$
      \EndIf
    \EndFor
    \If{$n_{0'} = n$}
      \Return набор пар $(i, j): M_{ij} = 0'$
    \EndIf
    \ForAll{столбцам из $M: \subset 0'$}
      выделить столбцы
    \EndFor
    \If{$\not\exists 0$ в выделенных}
      найти $m$ -- минимальный из выделенных
      \ForAll{невыделенным столбцам}
        вычесть $m$
      \EndFor
      \ForAll{выделенным строкам}
        прибавить $m$
      \EndFor
    \EndIf
    \If{$\exists 0$ в выделенных}
      пометить как $0'$
      \If{в одной строке с ним есть $0^*$}
        снять выделение со столбца с $0'$, выделить эту строку
      \EndIf
      построить L-цепочку
      $0^* \rightarrow 0', 0' \rightarrow 0^*$
      снять все выделения
    \EndIf
  \EndWhile
\end{algorithmic}

\section{Текст программы}

\lstinputlisting[language=Haskell]{../Main.hs}

\section{Индивидуальное задание}

Задание для данного варианта:

\[ \left[
\begin{matrix}
  4 & 10 & 10 & 3 & 6 \\
  5 & 6 & 2 & 7 & 4 \\
  9 & 5 & 6 & 8 & 3 \\
  2 & 3 & 5 & 4 & 8 \\
  8 & 5 & 4 & 9 & 3 \\
\end{matrix}
\right] \]

Для данного результата результаты работы следующие:

\begin{lstlisting}[language=Haskell]
> fst $ maximize test
[(1,3),(2,2),(3,1),(4,5),(5,4)]
> fst $ minimize test
[(1,4),(2,3),(3,5),(4,1),(5,2)]
\end{lstlisting}

\end{document}
