


% !TeX TS-program = XeLaTeX
\documentclass[a4paper,12pt]{article}\usepackage[]{graphicx}\usepackage[]{color}
%% maxwidth is the original width if it is less than linewidth
%% otherwise use linewidth (to make sure the graphics do not exceed the margin)
\makeatletter
\def\maxwidth{ %
  \ifdim\Gin@nat@width>\linewidth
    \linewidth
  \else
    \Gin@nat@width
  \fi
}
\makeatother

\definecolor{fgcolor}{rgb}{0.345, 0.345, 0.345}
\newcommand{\hlnum}[1]{\textcolor[rgb]{0.686,0.059,0.569}{#1}}%
\newcommand{\hlstr}[1]{\textcolor[rgb]{0.192,0.494,0.8}{#1}}%
\newcommand{\hlcom}[1]{\textcolor[rgb]{0.678,0.584,0.686}{\textit{#1}}}%
\newcommand{\hlopt}[1]{\textcolor[rgb]{0,0,0}{#1}}%
\newcommand{\hlstd}[1]{\textcolor[rgb]{0.345,0.345,0.345}{#1}}%
\newcommand{\hlkwa}[1]{\textcolor[rgb]{0.161,0.373,0.58}{\textbf{#1}}}%
\newcommand{\hlkwb}[1]{\textcolor[rgb]{0.69,0.353,0.396}{#1}}%
\newcommand{\hlkwc}[1]{\textcolor[rgb]{0.333,0.667,0.333}{#1}}%
\newcommand{\hlkwd}[1]{\textcolor[rgb]{0.737,0.353,0.396}{\textbf{#1}}}%

\usepackage{framed}
\makeatletter
\newenvironment{kframe}{%
 \def\at@end@of@kframe{}%
 \ifinner\ifhmode%
  \def\at@end@of@kframe{\end{minipage}}%
  \begin{minipage}{\columnwidth}%
 \fi\fi%
 \def\FrameCommand##1{\hskip\@totalleftmargin \hskip-\fboxsep
 \colorbox{shadecolor}{##1}\hskip-\fboxsep
     % There is no \\@totalrightmargin, so:
     \hskip-\linewidth \hskip-\@totalleftmargin \hskip\columnwidth}%
 \MakeFramed {\advance\hsize-\width
   \@totalleftmargin\z@ \linewidth\hsize
   \@setminipage}}%
 {\par\unskip\endMakeFramed%
 \at@end@of@kframe}
\makeatother

\definecolor{shadecolor}{rgb}{.97, .97, .97}
\definecolor{messagecolor}{rgb}{0, 0, 0}
\definecolor{warningcolor}{rgb}{1, 0, 1}
\definecolor{errorcolor}{rgb}{1, 0, 0}
\newenvironment{knitrout}{}{} % an empty environment to be redefined in TeX

\usepackage{alltt}

% polyglossia should go first!
\usepackage{polyglossia} % multi-language support
\setmainlanguage{russian}
\setotherlanguage{english}

\usepackage{amsmath} % math symbols, new environments and stuff
\usepackage{unicode-math} % for changing math font and unicode symbols
\usepackage[style=english]{csquotes} % fancy quoting
\usepackage{microtype} % for better font rendering
\usepackage[backend=biber]{biblatex} % for bibliography
\usepackage{hyperref} % for refs and URLs
\usepackage{graphicx} % for images (and title page)
\usepackage{geometry} % for margins in title page
\usepackage{tabu} % for tabulars (and title page)
\usepackage{placeins} % for float barriers
\usepackage{titlesec} % for section break hooks
%\usepackage[justification=centering]{caption} % for forced captions centering
%\usepackage{subcaption} % for subfloats
%\usepackage{rotating} % for rotated labels in tables
%\usepackage{tikz} % for TiKZ
%\usepackage{dot2texi} % for inline dot graphs
%\usepackage{listings} % for listings 
%\usepackage{upquote} % for good-looking quotes in source code (used for custom languages)
\usepackage{multirow} % for multirow cells in tabulars
%\usepackage{afterpage} % for nice landspace floats and longtabus
%\usepackage{pdflscape} % for landspace orientation
%\usepackage{xcolor} % colors!
%\usepackage{enumitem} % for unboxed description labels (long ones)
%\usepackage{numprint} % pretty-printing numbers
%\usepackage{longtable} % for longtabu

\defaultfontfeatures{Mapping=tex-text} % for converting "--" and "---"
\setmainfont{CMU Serif}
\setsansfont{CMU Sans Serif}
\setmonofont{CMU Typewriter Text}
\setmathfont{XITS Math}
\DeclareSymbolFont{letters}{\encodingdefault}{\rmdefault}{m}{it} % for Russian in math
%\MakeOuterQuote{"} % enable auto-quotation

% new page and barrier after section, also phantom section after clearpage for
% hyperref to get right page.
% clearpage also outputs all active floats:
%\newcommand{\sectionbreak}{\FloatBarrier\newpage\phantomsection}
%\newcommand{\subsectionbreak}{\FloatBarrier}
\renewcommand{\thesection}{\arabic{section}} % no chapters
\numberwithin{equation}{section}
\renewcommand{\figurename}{Рисунок} % Russian standards
%\usetikzlibrary{shapes,arrows,trees}

%\lstset{
%  numbers=left,
%  numberstyle=\scriptsize,
%  basicstyle=\ttfamily\scriptsize,
%  columns=fullflexible,
%  keepspaces, % for spaces in unicode text!
%  captionpos=t % Russian standards, again
%}
%\renewcommand{\lstlistingname}{Листинг}

\newcommand{\un}[1]{\: \mathit{#1}} % Unit measurements

\title{Организационно-экономический раздел}
\author{Амиантов Н.И., ИУ7-82}
\date{\today}

\makeatletter
\let\thetitle\@title
\let\theauthor\@author
\let\thedate\@date
\makeatother
\IfFileExists{upquote.sty}{\usepackage{upquote}}{}

\begin{document}

\maketitle

\section{Организация и планирование процесса разработки}
При использовании традиционного подхода, организация и планирование процесса
разработки программного продукта или программного комплекса предусматривает
выполнение следующих работ:
\begin{itemize}
\item формирование  состава  выполняемых  работ  и  группировка  их  по 
стадиям разработки;
\item расчет трудоемкости выполнения работ; 
\item установление  профессионального  состава  и  расчет  количества 
исполнителей; 
\item определение  продолжительности  выполнения  отдельных  этапов 
разработки; 
\item построение календарного графика выполнения разработки; 
\end{itemize}
 
Разработку  программного  продукта  можно  разделить  на  следующие 
стадии: 
\begin{description}
\item[Техническое  задание.] Постановка  задач.  Определение  состава  пакета 
прикладных  программ,  состава  и  структуры  информационной  базы.  Выбор 
языков  программирования.  Предварительный  выбор  методов  выполнения 
работы. Разработка календарного плана выполнения работ. 
\item[Эскизный  проект.] Предварительная  разработка  структуры  входных  и 
выходных  данных.  Разработка  общего  описания  алгоритмов  решения  задач. 
Разработка пояснительной записки. Консультации разработчиков постановки 
задач. Согласование и утверждение эскизного проекта. 
\item[Технический проект.] Разработка алгоритмов решения задач.  Разработка
пояснительной записки.  Согласование и утверждение технического проекта.
Разработка структуры программы.  Разработка программной документации и передача
ее для включения в технический проект.  Уточнение структуры, анализ и
определение формы представления входных и выходных данных. Выбор конфигурации
технических средств.
\item[Рабочий проект.] Комплексная отладка задач и сдача в опытную эксплуатацию.
Разработка проектной документации.  Программирование и отладка программ.
Описание контрольного примера.  Разработка программной документации.
Разработка, согласование программы и методики испытаний. Предварительное
проведение всех видов испытаний.
\item[Внедрение.] Подготовка и передача программной документации для
сопровождения с оформлением соответствующего акта.  Передача программной
продукции в фонд алгоритмов и программ.  Проверка алгоритмов и программ решения
задач, корректировка документации после опытной эксплуат ции программного
продукта.
\end{description}
Планирование длительности этапов и содержания проекта осуществляется в
соответствии с ЕСПД ГОСТ 34.603-92 и распределяет работы по этапам, как показано
в таб.~\ref{tab:stages}.

\begin{table}
  \begin{tabu} {|X[-1]|l|X|} \hline
    Основные стадии & № & Содержание \\\hline
    \multirow{2}{*}{Техническое задание}
    & 1 & Постановка задачи \\\cline{2-3}
    & 2 & Выбор средств разработки и реализации \\\hline
    \multirow{4}{*}{Эскизный проект}
    & 3 & Разработка структурной схемы системы \\\cline{2-3}
    & 4 & Разработка структуры базы данных \\\cline{2-3}
    & 5 & Разработка алгоритмов доступа к данным \\\cline{2-3}
    & 6 & Разработка алгоритмов анализа данных \\\hline
    \multirow{8}{*}{Техно-рабочий проект}
    & 7 & Реализация алгоритмов доступа к данным \\\cline{2-3}
    & 8 & Реализация алгоритмов анализа данных \\\cline{2-3}
    & 9 & Разработка пользовательского интерфейса для обслуживающего персонала \\\cline{2-3}
    & 10 & Разработка  пользовательского  интерфейса  для клиента \\\cline{2-3}
    & 11 & Реализация  пользовательского  интерфейса  для клиента \\\cline{2-3}
    & 12 & Тестирование и отладка программного комплекса \\\cline{2-3}
    & 13 & Разработка документации к системе \\\cline{2-3}
    & 14 & Итоговое тестирование системы \\\hline
    {Внедрение}
    & 15 & Установка и настройка ПП \\\hline
  \end{tabu}
  \label{tab:stages}
  \caption{Распределение работ проекта по этапам.}
\end{table}

\section{Расчет трудоемкости выполнения работ}
Трудоемкость  разработки  программной  продукции  зависит  от  ряда 
факторов, основными из которых являются следующие:
\begin{itemize}
\item степень новизны разрабатываемого программного комплекса,  
\item сложность алгоритма его функционирования,  
\item объем используемой информации, вид ее представления и способ обработки,
\item уровень используемого алгоритмического языка программирования
\end{itemize}
Исходные данные расчета представлены в таб.~\ref{tab:source}.

\begin{table}
  \begin{tabu} {|X[-1]|X|} \hline
    Функциональное назначение ПП & Задачи расчётного характера \\\hline
    Степень новизны разрабатываемого проекта & Группа новизны В -- продукт, имеющий аналоги \\\hline
    Степень сложности алгоритма функционирования & 2 группа сложности --
    программная продукция, реализующая учётно-статистические алгоритмы \\\hline
    По виду представления исходной информации & Группа 12 -- исходная информация
    представлена в форме документов, имеющих одинаковый формат и структуру,
    требуется форматный контроль информации. \\\hline
    Структура выходных документов & Группа 22 -- требуется вывод на печать одинаковых
    документов, вывод информационных массивов на машинные носители. \\\hline
  \end{tabu}
  \label{tab:source}
  \caption{Исходные данные}
\end{table}

Трудоемкость  разработки  программной  продукции $\tau_{ПП}$ может  быть 
определена  как  сумма  величин  трудоемкости  выполнения отдельных стадий 
разработки ПП из выражения:
\begin{equation} \label{eq:diff}
  \tau_{ПП} = \tau_{ТЗ} + \tau_{ЭП} + \tau_{ТП} + \tau_{РП} + \tau_{В}
\end{equation}
, где
\begin{itemize}
\item $\tau_{ТЗ}$ -- трудоемкость разработки технического задания на создание ПП; 
\item $\tau_{ЭП}$ -- трудоемкость разработки эскизного проекта ПП; 
\item $\tau_{ТП}$ -- трудоемкость разработки технического проекта ПП; 
\item $\tau_{РП}$ -- трудоемкость разработки рабочего проекта ПП; 
\item $\tau_{В}$ -- трудоемкость внедрения разработанного ПП. 
\end{itemize}
  
Трудоемкость  разработки  технического  задания  рассчитывается  по 
формуле:
\begin{equation}
  \tau_{ТЗ} = T_{РЗ}^З + T_{РП}^З
\end{equation}
, где
\begin{itemize}
\item $T_{РЗ}^З$ -- затраты времени разработчика постановки задач на разработку
  ТЗ, чел.  дни;
\item $T_{РП}^З$ -- затраты времени разработчика программного обеспечения на
  разработку ТЗ, чел. дни.
\end{itemize}
Значения величин $Т_{РЗ}^З$ и $Т_{РП}^З$ рассчитываются по формулам:
\begin{align*}
  T_{РЗ}^З &= t_З \cdot K_{РЗ} \\
  T_{РП}^З &= t_З \cdot K_{РП}
\end{align*}
, где:
\begin{description}
\item[$t_з$] -- норма времени на разработку ТЗ на программный продукт в
  зависимости от функционального назначения и степени новизны разрабатываемого ПП,
  чел. дни;
\item[$K_{РЗ}^З$] -- коэффициент, учитывающий удельный вес трудоемкости работ,
  выполняемых разработчиком постановки на стадии ТЗ;
\item[$K_{РП}^З$] -- коэффициент, учитывающий удельный вес трудоемкости работ,
  выполняемых разработчиком программного обеспечения на стадии ТЗ.
\end{description}

\begin{align*}
  t_З &= 47 \un{чел.дн.} \\
  K_{РЗ}^З &= 0.6 \\
  K_{РП}^З &= 1 \\
  \tau_{ТЗ} &= 75.2 \un{чел.дн.}
\end{align*}

Аналогично рассчитывается трудоемкость эскизного проекта ПП $\tau_{ЭП}$:
\begin{align*}
  t_{ЭП} &= 60 \\
  K_{РЗ}^Э &= 0.6 \\
  K_{РП}^Э &= 0.4 \\
  \tau_{ЭП} &= 47 \un{чел.дн.}
\end{align*}

Трудоемкость разработки технического проекта $\tau_{ТП}$ зависит от
функционального назначения ПП, количества разновидностей форм входной и выходной
информации и определяется как сумма времени, затраченного разработчиком
постановки задач и разработчиком программного обеспечения, т.е.
\begin{equation}
  \tau_{ТП} = \left( t_{РЗ}^Т + t_{РП}^Т \right) \cdot K_В \cdot K_Р
\end{equation}
, где
\begin{description}
\item[$t_{РЗ}^Т$, $t_{РП}^Т$] -- норма времени, затрачиваемого на разработку
технического проекта (ТП) разработчиком постановки задач и разработчиком
программного обеспечения соответственно, чел. дни;
\item[$K_В$] -- коэффициент учета вида используемой информации; 
\item[$K_Р$] -- коэффициент учета режима обработки информации. 
\end{description}
Значение коэффициента KВ определяется из выражения:
\begin{equation}
  K_В = \left( K_П \cdot n_П + K_{НС} \cdot n_{НС} + K_Б \cdot n_Б \right) / \left( n_П + n_{НС} + n_Б \right) 
\end{equation}
, где
\begin{description}
  \item[$K_П$, $K_{НС}$, $K_Б$] -- значения коэффициентов учета вида
используемой информации для переменной, нормативно-справочной информации и баз
данных соответственно;
\item[$n_П$, $n_{НС}$, $n_Б$] -- количество наборов данных переменной,
нормативно-справочной информации и баз данных соответственно.
\end{description}
\begin{align*}
  n_п &= 1 \\
  n_{нс} &= 0 \\
  n_б &= 0 \\
  K_Р &= 1.26 \\
  K_П &= 1 \\
  K_{НС} &= 1 \\
  K_Б &= 1 \\
  K_В &= 1 \\
  t_{РЗ}^Т &= 48 \\
  t_{РП}^Т &= 12 \\
  \tau_{ТП} &= \left( t_{РЗ}^Т + t_{РП}^Т \right) \cdot K_В \cdot K_Р =
  75.6 \un{чел.дн.}
\end{align*}
Трудоемкость разработки рабочего проекта $\tau_{РП}$ зависит от функционального
назначения ПП, количества разновидностей форм входной и выходной информации,
сложности алгоритма функционирования, сложности контроля информации, степени
использования готовых программных модулей, уровня алгоритмического языка
программирования и определяется по формуле:
\begin{equation}
  \tau_{РП} = K_К \cdot K_Р \cdot K_Я \cdot K_З \cdot K_{ИА} \cdot \left( t_{РЗ}^Р + t_{РП}^Р \right)
\end{equation}
, где
\begin{description}
\item[$K_К$] -- коэффициент учета сложности контроля информации; 
\item[$K_Я$] -- коэффициент  учета  уровня  используемого  алгоритмического 
языка программирования; 
\item[$K_З$] -- коэффициент учета степени использования готовых программных 
модулей;
\item[$K_{ИА}$] -- коэффициент учета вида используемой информации и сложности 
алгоритма ПП;
\item[$t_{РЗ}^Р$, $t_{РП}^Р$] -- норма времени, затраченного на разработку РП на
алгоритмическом языке высокого уровня разработчиком постановки задач и
разработчиком программного обеспечения соответственно, чел. дни.
\end{description}
Значение коэффициента $K_{ИА}$ определяется из выражения
\begin{equation}
  K_{ИА} = \left( K_П' \cdot n_П + K{НС}' \cdot n_{НС} + K_Б' \cdot n_Б \right) / \left( n_П + n_{НС} + n_Б \right)
\end{equation}
где 
\begin{description}
\item[$K_П'$, $K_{НС}'$, $K_Б'$] -- значения коэффициентов учета сложности
алгоритма ПП и вида используемой информации для переменной,
нормативно-справочной информации и баз данных соответственно.
\end{description}
\begin{align*}
  K_К &= 1 \\
  K_Я &= 1 \\
  K_З &= 0.5 \\
  t_{РЗ}^Р &= 10 \un{чел.дн.} \\
  t_{РП}^Р &= 44 \un{чел.дн.} \\
  K_П' &= 0.4 \\
  K_{НС}' &= 0.4 \\
  K_Б' &= 0.4 \\
  K_{ИА} &= 0.4 \\
  \tau_{РП} &= 13.608 \un{чел.дн.}
\end{align*}
Так как при разработке ПП стадии ``Технический проект'' и ``Рабочий проект''
объединены в стадию ``Техно-рабочий проект'', то трудоемкость ее выполнения
$\tau_{ТРП}$ определяется по формуле:
\begin{align*}
  \tau_{ТРП} = 0.85 \cdot \tau_{ТП} + \tau_{РП} = 77.868 \un{чел.дн.}
\end{align*}
Трудоемкость выполнения стадии внедрения $\tau_{В}$ может быть рассчитана по
формуле:
\begin{equation}
  \tau_В = \left( t_{РЗ}^В + t_{РП}^В \right) \cdot K_К \cdot K_Р \cdot K_З
\end{equation}
, где
\begin{description}
\item[$t_{РЗ}^В$, $t_{РП}^В$] -- норма времени, затрачиваемого разработчиком
постановки задач и разработчиком программного обеспечения соответственно на
выполнение процедур внедрения ПП, чел.дни.
\end{description}
\begin{align*}
  t_{РЗ}^В &= 11 \un{чел.дн.} \\
  t_{РП}^В &= 12 \un{чел.дн.} \\
  \tau_В &= 14.49 \un{чел.дн.}
\end{align*}
Подставляя полученные данные в~\ref{eq:diff}, получим:
\begin{equation}
  \tau_{ПП} = 225.898 \un{чел.дн.}
\end{equation}

В таб.~\ref{tab:diff} представлены трудоемкости по этапам разработки проекта. 




\begin{table}
  \begin{tabu} {|X|X|l|X|X|} \hline
    Этап & Общая труд-ть & № & Содержание & Трудоемкость, чел-дн. \\\hline
\multirow{2}{*}{1 (ТЗ)} & \multirow{2}{*}{75.200000} 
& 1 & Постановка задачи & 75.200000 \\\cline{3-5}
 & & 2 & Выбор средств разработки и реализации & 0.000000 \\\cline{3-5}
\hline
\multirow{4}{*}{2 (ЭП)} & \multirow{4}{*}{47.000000} 
& 3 & Разработка структурной схемы системы & 47.000000 \\\cline{3-5}
 & & 4 & Разработка структуры базы данных & 0.000000 \\\cline{3-5}
 & & 5 & Разработка алгоритмов доступа к данным & 0.000000 \\\cline{3-5}
 & & 6 & Разработка алгоритмов анализа данных & 0.000000 \\\cline{3-5}
\hline
\multirow{6}{*}{3 (ТП, РП)} & \multirow{6}{*}{89.208000} 
& 7 & Разработка пользовательского интерфейса для обслуживающего персонала & 89.208000 \\\cline{3-5}
 & & 8 & Разработка пользовательского интерфейса для клиента & 0.000000 \\\cline{3-5}
 & & 9 & Реализация пользовательского интерфейса для клиента & 0.000000 \\\cline{3-5}
 & & 10 & Тестирование и отладка программного комплекса & 0.000000 \\\cline{3-5}
 & & 11 & Разработка документации к системе & 0.000000 \\\cline{3-5}
 & & 12 & Итоговое тестирование системы & 0.000000 \\\cline{3-5}
\hline
\multirow{1}{*}{4 (В)} & \multirow{1}{*}{14.490000} 
& 13 & Установка и настройка ПП & 14.490000 \\\cline{3-5}
\hline


    Итого & & & & 225.898 \\\hline
  \end{tabu}
  \label{tab:diffs}
  \caption{Трудоемкости по стадиям разработки проекта}
\end{table}

Средняя численность исполнителей при реализации проекта разработки 
и внедрения ПО определяется соотношением $$N = \frac{Q_p}{F}$$, где:
\begin{description}
\item[$Q_p$] -- затраты  труда  на  выполнение  проекта  (разработка  и  внедрение 
ПО),
\item[$F$] -- фонд рабочего времени.
\end{description}

Величина фонда рабочего времени определяется соотношением: 
\begin{equation}
  F = T \cdot F_М
\end{equation}
, где
\begin{description}
\item[$Т$] -- время выполнения проекта в месяцах. $T = 5 \un{мес.}$; 
\item[$F_М$] -- фонд времени в текущем месяце, который рассчитывается из учета 
  общества числа дней в году, числа выходных и праздничных дней: 
  \begin{equation}
    F_М = \frac{t_р \cdot \left( D_К - D_З - D_П \right)}{12}
  \end{equation}
  , где
  \begin{description}
  \item[$t_р = 8$] -- продолжительность рабочего дня; 
  \item[$D_K = 365$] -- общее число дней в году; 
  \item[$D_B = 103$] -- число выходных дней в году; 
  \item[$D_П = 13$] -- число праздничных дней в году. 
  \end{description}
\end{description}

\begin{align*}
  F_М &= 1992 \\
  F &= T \cdot F_М = 9960 \\
  N &= \frac{t_р \cdot \tau_{ПП}}{F} = 1 \:\text{(число исполнителей проекта)}
\end{align*}
 
\section{Календарный план-график разработки ПП}
Планирование и контроль хода выполнения разработки проводится по календарному
графику выполнения работ.  Планирование процесса разработки и календарный
ленточный план представлены в таб.~\ref{tab:plan} и \ref{fig:plan} соответственно.




\begin{table}
  \begin{tabu} {|l|l|X|l|l|} \hline
    Стадия & Трудоемкость & Должность & Распределение & Кол-во \\\hline
\multirow{1}{*}{1. ТЗ} & \multirow{1}{*}{75.200000} 
& Ведущий инженер & 75.200000 & 1 \\
\hline
\multirow{1}{*}{2. ЭП} & \multirow{1}{*}{47.000000} 
& Ведущий инженер & 47.000000 & 1 \\
\hline
\multirow{1}{*}{3. ТП, ЭП} & \multirow{1}{*}{89.208000} 
& Ведущий инженер & 89.208000 & 1 \\
\hline
\multirow{1}{*}{4. В} & \multirow{1}{*}{14.490000} 
& Ведущий инженер & 14.490000 & 1 \\
\hline


    Итого: & 225.898 & & & 1 \\\hline
  \end{tabu}
  \label{tab:plan}
  \caption{Планирование процесса разработки}
\end{table}

Вывод:  при  распараллеливании  работы  ведущего  инженера  и 
программиста  можно  добиться  сокращения  срока  разработки  и  внедрения 
программного продукта с 194 дней до 106 дней, т. е. в 1.73 раза по сравнению 
со  временем  разработки  одним  человеком,  что  близко  к  теоретическому 
значению.

\section{Расчёт стоимости программного продукта}

Затраты на выполнение проекта состоят из затрат на заработную плату 
исполнителям,  затрат  на  закупку  или  аренду  оборудования,  затрат  на 
организацию рабочих мест, и затрат на накладные расходы.

\end{document}
