% !TeX TS-program = XeLaTeX
% check for outdated packages, syntax etc.
\RequirePackage[l2tabu,orthodox]{nag}

\documentclass[a4paper,12pt]{article}

% polyglossia should go first!
\usepackage{polyglossia} % multi-language support
 
\usepackage{amsmath} % math symbols, new environments and stuff
\usepackage{unicode-math} % for changing math font and unicode symbols
\usepackage[style=english]{csquotes} % fancy quoting
\usepackage{microtype} % for better font rendering
\usepackage[backend=biber]{biblatex} % for bibliography
\usepackage{hyperref} % for refs and URLs
\usepackage{graphicx} % for images (and title page)
\usepackage{geometry} % for margins in title page
\usepackage{tabu} % for tabulars (and title page)
\usepackage{placeins} % for float barriers
\usepackage{titlesec} % for section break hooks
% \usepackage{cleveref} % ref auto-naming (doesn't support polyglossia)
%\usepackage[justification=centering]{caption} % for forced captions centering
%\usepackage{subcaption} % for subfloats
%\usepackage{rotating} % for rotated labels in tables
%\usepackage{tikz} % for TiKZ
%\usepackage{dot2texi} % for inline dot graphs
%\usepackage{listings} % for listings 
%\usepackage{upquote} % for good-looking quotes in source code (used for custom languages)
%\usepackage{multirow} % for multirow cells in tabulars
%\usepackage{afterpage} % for nice landspace floats and longtabus
%\usepackage{pdflscape} % for landspace orientation
%\usepackage{xcolor} % colors!
%\usepackage{enumitem} % for unboxed description labels (long ones)
%\usepackage{numprint} % pretty-printing numbers
%\usepackage{longtable} % for longtabu

\setmainlanguage{russian}
\setotherlanguage{english}
\defaultfontfeatures{Mapping=tex-text} % for converting "--" and "---"
\setmainfont{CMU Serif}
\setsansfont{CMU Sans Serif}
\setmonofont{CMU Typewriter Text}
\setmathfont{XITS Math}
\DeclareSymbolFont{letters}{\encodingdefault}{\rmdefault}{m}{it} % for Russian in math
%\MakeOuterQuote{"} % enable auto-quotation

%\newcommand{\un}[1]{\: \mathit{#1}} % Unit measurements

\makeatletter
\let\thetitle\@title
\let\theauthor\@author
\let\thedate\@date
\makeatother

\addglobalbib{summary.bib}

\title{Диплом}
\author{Амиантов Н.И., ИУ7-82}
\date{\today}

\begin{document}
\pagenumbering{gobble}

\begin{center}
{\Large Рецензия} \\
на квалификационную работу на степень бакалавра по специальности \\
``Программное обеспечение ЭВМ и информационные технологии'' \\
бакалавра группы ИУ7-82 МГТУ им. Н. Э. Баумана \\
Амиантова Николая Ильича на тему \\
``Разработка алгоритма сжатия изображений, формата для хранения и сопутствующих утилит''
\end{center}

Рецензируемая бакалаврская квалификационная работа посвящена разработке формата
хранения изображений, алгоритмов сжатия изображений с потерями с преобразованием
их в разработанный формат и разработке сопутствующей утилиты для упаковки и
распаковки. Сжатие изображений является важной областью, инновации в которой
востребованы в настоящее время в связи с возрастающими объёмами данных для
хранения. При разработке нового алгоритма было проведено глубокое изучение
предметной области, проанализированы известные алгоритмы, форматы, их
возможности, достоинства и недостатки. Пояснительная записка к бакалаврской
квалификационной работе состоит ил технического задания, календарного плана,
введения, четырёх основных частей, заключения, списка литературы.  В
аналитической части произведена постановка задачи, выполнен анализ предметной
области и изучение известных алгоритмов и форматов хранения изображений. В
конструкторской части приведено описание формата хранения изображений и общих
частей алгоритма сжатия, а также описание алгоритма распаковки. В
технологической части приведены обоснование выбранных средств и технологий
разработки, диаграммы классов, требования к архитектуре и описание процесса
тестирования разработанного программного комплекса. В исследовательской части
проводится поиск и сравнение подходящих собственных алгоритмов для основных
частей алгоритма сжатия, сравниваются различные подходы к обработке изображения
и делаются выводы об оптимальных параметрах сжатия. К недостаткам
квалификационной работы можно отнести отсутствие исследования алгоритмической
структуры предложенных методов, однако данный недостаток не снижает ценности
дипломной работы. Объём работ выполнен согласно техническому заданию и полностью
удовлетворяет квалификационным требованиям, предъявляемым к выпускным
бакалаврским работам. Работа заслуживает оценки ``отлично'', а её автор, Амиантов
Николай Ильич, -– присвоения степени бакалавра по специальности 230100.

\vfill

Доцент кафедры ИУ6, к.т.н. \hfill /Попов А.Ю./

\end{document}